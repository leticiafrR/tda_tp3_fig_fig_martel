\section{Conclusiones}
Hasta este punto hemos logrado observar que la solución de nuestro problema global se compone de subsoluciones que en su determinado minuto fueron las óptimas; inclusive, encontramos la forma en cómo se componen dichas subsoluciones para dar respuesta a problemas más grandes de la misma índole. Dichas caracterisiticas adicionadas a que la problemática nos ofrece un orden natural en el que los hechos se desarrollan o en el que podemos agrandar nuestro problema y que la cantidad de estos mismos sea discreta nos permite plantear una solución mediante programación dinámica. De todas formas, el identificar inicialmente que posiblemente se cumplan estas características en el problema no resta importacia ni complejidad al proceso de entender cómo es que estas caracaterísticas están presentes en el problema ni en cómo se concretizarían teóricamente en la ecuación de recurrencia.



También encontramos relevante el mencionar que, si bien resultó intrincado el identificar y concretizar asertadamente la forma y composición del problema en la ecuación de recurrencia, una vez hecho esto el proceso de la programación propiamente dicha lo consideramos una parte relativamente sencilla, lo que no quita que en este proceso surjan cuestiones importantes a definir, como qué información es imprescindible para resolver correctamente el problema y reconstruir la solucion, que viene estrechamente relacionado con qué datos se deben memorizar, optimizaciones que podamos adaptar y los casos en los que se efectuan estas (como las estudiadas en el desarrollo del presente informe). 