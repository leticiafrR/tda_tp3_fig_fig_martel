\section{Introducción}

La finalidad del presente trabajo práctico consiste en el planteamiento y análisis de un algoritmo que logre determinar cuál sería la \textit{mejor estrategia} de ataque para la policía secreta de Ba Sing Se, los Dai Li, frente al \textit{ataque ráfaga} que la nación del fuego efectuará. 

Para el plantemiento de dicho algoritmo resulta imprescindible llevar a cabo un análisis exhaustivo del problema mismo: su forma y composición, y finalmente la ecuación de recurrencia del mismo. 

Una vez consolidada nuestra propuesta de algoritmo, expondremos el análisis correspondiente al mismo, considerando factores de nuestro interés como la complejidad temporal y espacial del mismo, así como los efectos de la variabilidad de los valores dados sobre los tiempos de ejecución y sobre la optimalidad de la solución a la que llegamos.\\



\textbf{Sobre la estrategia a obtener:}
{\setlength{\parindent}{25pt}
\begin{itemize}
    \item El criterio para considerar una estrategia como $mejor$ se basa en que la misma \textbf{maximice la cantidad de bajas totales} en las tropas enemigas.
    \item  La estrategia resultante debe indicar en qué minutos resulta más conveniente llevar a cabo un ataque (empleando el total de energía disponibe hasta ese momento) así como cuándo no atacar, es decir, usar ese tiempo para cargar la energía disponible para el próximo ataque.\\
\end{itemize}
}


\textbf{Sobre la información disponible:}
{\setlength{\parindent}{25pt}

\begin{itemize}
    \item El \textit{ataque ráfaga} que la nación del fuego planea llevar a cabo consiste en una susesión de hordas de soldados aproximándose cada minuto, misma que durará $n$ minutos.
    \item Es conocida la cantidad de soldados con los que cuenta cada horda, en cada minuto $i\in\mathbb{N}/ \\ i\leq n$. Esta cantidad la representamos como $x_i$.
    \item También conocemos qué intesidad tiene un ataque de los Dai Li, es decir, cuántas bajas podría generar en las tropas enemigas (ya que de no haber soldados, aún si se tiene una intensidad alta , no se genera ninguna baja) respecto al tiempo de carga de energía. Esta cantidad la representamos como $f_i$
\end{itemize}

}

